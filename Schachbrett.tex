\chapter{Programmierung eines Schachbrettes}

\begin{center}
\begin{pspicture*}[showgrid=false](8.0,8.0)
    \multido{\iLowE=1+2, \iHighE=2+2, \iLowO=0+2, \iHighO=1+2, \iFarbeG=25+25, \iFarbeU=12+25}{4}
    {
        % Zeichnen der Reihen 2,4,6,8
        \multido{\iA=2+2,\iB=1+2}{4}{\psframe[linecolor=Black100,fillstyle=solid,fillcolor=blue!\iFarbeG,linewidth=0pt](\iA,\iLowE)(\iB,\iHighE)} %n. Reihe schwärze Kästen
        \multido{\iA=0+2,\iB=1+2}{4}{\psframe[linecolor=Black100,fillstyle=solid,fillcolor=White100,linewidth=0pt](\iA,\iLowE)(\iB,\iHighE)} %n. Reihe weisse Kästen
        % Zeichnen der Reihen 1,3,6,7
        \multido{\iA=0+2,\iB=1+2}{4}{\psframe[linecolor=Black100,fillstyle=solid,fillcolor=blue!\iFarbeU,linewidth=0pt](\iA,\iLowO)(\iB,\iHighO)} %n. Reihe gestufte graue Kästen
        \multido{\iA=2+2,\iB=1+2}{4}{\psframe[linecolor=Black100,fillstyle=solid,fillcolor=White100,linewidth=0pt](\iA,\iLowO)(\iB,\iHighO)} %n. Reihe weisse Kästen
    }
\end{pspicture*}
\end{center}